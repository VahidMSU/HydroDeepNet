%% Supplementary Material for Environmental Modeling and Software
%% Using Elsevier's document class `elsarticle' with numbered style bibliographic references
%%
\documentclass[preprint,12pt]{elsarticle}

%% Use the option review to obtain double line spacing
%% \documentclass[preprint,review,12pt]{elsarticle}

%% For final submission:
%% \documentclass[final,3p,times]{elsarticle}

%% The amssymb package provides various useful mathematical symbols
\usepackage{amssymb}
%% The amsmath package provides various useful equation environments
\usepackage{amsmath}
%% The amsthm package provides extended theorem environments
\usepackage{amsthm}
%% The lineno packages adds line numbers
\usepackage{lineno}
\usepackage{graphicx}
\usepackage{hyperref}
\usepackage{booktabs}
\usepackage{float}
\usepackage{url}
\usepackage{appendix}

\journal{Environmental Modeling and Software}

\begin{document}

\begin{frontmatter}

\title{Supplementary Material for: SWATGenX: An automated Web application platform for generating SWAT+ with NHDPlus HR across CONUS}

\author[1]{Author One\corref{cor1}}
\author[2]{Author Two}
\author[1,3]{Author Three}
\cortext[cor1]{Corresponding author}
\ead{email@address.com}
\address[1]{Department, University, City, State/Province, Country}
\address[2]{Department, University, City, State/Province, Country}
\address[3]{Department, University, City, State/Province, Country}

\begin{abstract}
This document provides supplementary information, data, and analyses that support the main manuscript but are not essential to the primary results and conclusions of our paper "SWATGenX: An automated Web application platform for generating SWAT+ with NHDPlus HR across CONUS".
\end{abstract}

\begin{keyword}
SWAT+ \sep groundwater modeling \sep supplementary material \sep CONUS \sep web application \sep NHDPlus HR
\end{keyword}

\end{frontmatter}

%% Uncomment to enable line numbers
%% \linenumbers

\section*{Overview of Supplementary Material}
\label{sec:overview}
This document provides supplementary information, data, and analyses that support the main manuscript but are not essential to the primary results and conclusions. The material is organized to correspond with the structure of the main paper.

\section{Supplementary Methods}
\label{sec:suppl_methods}

\subsection{Detailed data preprocessing workflow}
\label{subsec:preprocessing}
% Additional details on data preprocessing steps

\subsection{Model parameterization}
\label{subsec:parameterization}
% Details of model parameters and configurations

\subsection{Technical implementation details}
\label{subsec:tech_implementation}
% Software architecture, database design, etc.

\section{Supplementary Results}
\label{sec:suppl_results}

\subsection{Additional performance metrics}
\label{subsec:add_metrics}
% Additional metrics not included in the main manuscript

\subsection{Regional comparison of model performance}
\label{subsec:regional_comparison}
% Detailed comparison across different regions

\subsection{Sensitivity analysis}
\label{subsec:sensitivity}
% Results of sensitivity analysis for model parameters

\section{Additional Figures and Tables}
\label{sec:add_figures}
% Additional visual materials supporting the manuscript

\section{Case Studies}
\label{sec:case_studies}

\subsection{Case study 1: Detailed Michigan watershed analysis}
\label{subsec:michigan_case}
% In-depth analysis of a specific watershed

\subsection{Case study 2: Cross-regional comparison}
\label{subsec:cross_regional}
% Comparative analysis across different regions

\section{Web Application Documentation}
\label{sec:web_app_doc}

\subsection{API documentation}
\label{subsec:api_doc}
% Details of API endpoints and usage

\subsection{User interface description}
\label{subsec:ui_description}
% Screenshots and description of the web interface

\subsection{Code examples}
\label{subsec:code_examples}
% Example code for accessing and using the platform

\bibliographystyle{elsarticle-num}
\bibliography{references}

%% If you prefer to manually include references, use:
%% \begin{thebibliography}{00}
%% \bibitem{key}
%%   Author,
%%   \textit{Title},
%%   Journal,
%%   Year.
%% \end{thebibliography}

\end{document}
